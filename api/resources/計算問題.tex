\documentclass[uplatex,a4j,11pt]{jsreport}
\usepackage[margin=10mm,includefoot]{geometry}
\usepackage{emath}
\usepackage{bxpapersize}
\usepackage{multicol}
\usepackage{enumitem}

\setlist[enumerate]{
  leftmargin=*,
  itemindent=0pt,
  itemsep=1em,
  labelsep=1em,
  label=(\arabic*)}

\pagestyle{empty}

\begin{document}

%%% 単元設定
%%% 書式:{学年} {コード} {単元名}
%define 中1 0100 正負の二項計算
%define 中1 0101 正負の加法・減法(自動生成)
%define 中1 0102 正負の乗法・除法(自動生成)
%define 中1 0103 指数(自動生成)
%define 中1 0104 指数を含む二項計算(自動生成)
%define 中1 0105 正負の四則混合(自動生成)
%define 中1 0110 文字式の表し方
%define 中1 0115 式の値
%define 中1 0116 文字式の計算
%define 中1 0120 一次方程式
%define 中1 0125 一次方程式の利用
%define 中1 0130 比例
%define 中1 0135 反比例
%define 中1 0140 平面図形
%define 中1 0145 空間図形
%define 中1 0150 データ
%define 中2 0200 文字式
%define 中2 0201 多項式の計算(自動生成)
%define 中2 0205 等式変形
%define 中2 0210 連立方程式
%define 中2 0215 連立方程式の利用
%define 中2 0220 一次関数
%define 中2 0230 図形
%define 中2 0240 確率
%define 中3 0300 式の展開
%define 中3 0305 因数分解
%define 中3 0310 平方根
%define 中3 0320 二次方程式
%define 中3 0330 二次関数
\begin{multicols}{2}
\begin{enumerate}
    %% H25後期
    \item $7+(-9)$ を計算しなさい。%% H25後期, 0100
    \item $-2$
    \item $8\div\left(-\bunsuu{2}{3}\right)^2$ を計算しなさい。%% H25後期, 0100
    \item $18$
    \item $3(2a-b)-(a-3b)$ を計算しなさい。%% H25後期, 0200
    \item $5a$
    \item 正十角形の1つの内角の大きさを求めなさい。%% H25後期, 0230
    \item $144^\circ$
    \item $(\sqrt{5}+\sqrt{3})(2\sqrt{5}-\sqrt{3})$ を計算しなさい。%% H25後期, 0310
    \item $7+\sqrt{15}$
    \item 二次方程式 $x^2-3x-1=0$ を解きなさい。%% H25後期, 0320
    \item $x=\bunsuu{3\pm\sqrt{13}}{2}$
    %% H25前期
    \item $7-(-5)$ を計算しなさい。%% H25前期, 0100
    \item $12$
    \item $(-4)^2+3\times(-2)$ を計算しなさい。%% H25前期, 0105
    \item $10$
    \item $\bunsuu{3}{2}x-6y-\bunsuu{1}{4}(3x-8y)$ を計算しなさい。%% H25前期, 0200
    \item $\bunsuu{3}{4}x-4y$
    \item 比例式 $2:5=(x-2):(x+7)$ をみたす $x$ の値を求めなさい。%% H25前期, 0120
    \item $x=8$
    \item $\sqrt{45}-\sqrt{20}+\bunsuu{15}{\sqrt{5}}$ を計算しなさい。%% H25前期, 0310
    \item $4\sqrt{5}$
    \item $(x+1)(x-7)-20$ を因数分解しなさい。%% H25前期, 0305
    \item $(x-9)(x+3)$
    %% H26後期
    \item $-3+8$ を計算しなさい。%% H26後期, 0100
    \item $5$
    \item $7-(-6)^2\div9$ を計算しなさい。%% H26後期, 0105
    \item $3$
    \item $6\left(\bunsuu{1}{3}a-\bunsuu{1}{2}b\right)-2a+b$ を計算しなさい。%% H26後期, 0200
    \item $-2b$
    \item $\sqrt{8}\times(-\sqrt{5})$ を計算しなさい。%% H26後期, 0310
    \item $-2\sqrt{10}$
    \item $x=-2$ のとき,$x^2+\bunsuu{10}{x}$ の値を求めなさい。%% H26後期, 0115
    \item $-1$
    \item $x^2-x-12$ を因数分解しなさい。%% H26後期, 0305
    \item $(x+3)(x-4)$
    %% H26前期
    \item $8-(-13)$ を計算しなさい。%% H26前期, 0100
    \item $21$
    \item $(-3)^2+\left(-\bunsuu{1}{3}\right)\times6$ を計算しなさい。%% H26前期, 0105
    \item $7$
    \item $(7a-4b)+\bunsuu{1}{2}(2b-6a)$ を計算しなさい。%% H26前期, 0200
    \item $4a-3b$
    \item 方程式 $0.2(x-2)=x+1.2$ を解きなさい。%% H26前期, 0120
    \item $x = -2$
    \item $\sqrt{48}-\sqrt{27}+5\sqrt{3}$ を計算しなさい。%% H26前期, 0310
    \item $6\sqrt{3}$
    \item 二次方程式 $x^2+7x+5=0$ を解きなさい。%% H26前期, 0320
    \item $x=\bunsuu{-7\pm\sqrt{29}}{2}$
    %% H27後期
    \item $11+(-8)$ を計算しなさい。%% H27後期, 0100
    \item $3$
    \item $2^2\div\left(-\bunsuu{2}{5}\right)$ を計算しなさい。%% H27後期, 0105
    \item $-10$
    \item $2(3x+y)-3(x-2y)$ を計算しなさい。%% H27後期, 0200
    \item $3x+8y$
    \item 方程式 $\bunsuu{3x-1}{2}=\bunsuu{5x+2}{3}$ を解きなさい。%% H27後期, 0120
    \item $x=-7$
    \item $(\sqrt{7}-\sqrt{5})(\sqrt{7}+2\sqrt{5})$ を計算しなさい。%% H27後期, 0310
    \item $-3+\sqrt{35}$
    \item $(x-1)^2-3(x-1)+2$ を因数分解しなさい。%% H27後期, 0305
    \item $(x-2)(x-3)$
    %% H27前期
    \item $(-7)-(-4)$ を計算しなさい。%% H27前期, 0100
    \item $-3$
    \item $(-4)^2+8\div(-2)$ を計算しなさい。%% H27前期, 0105
    \item $12$
    \item $\bunsuu{1}{2}(3a-2b)-(2a-b)$ を計算しなさい。%% H27前期, 0200
    \item $-\bunsuu{1}{2}a$
    \item 等式 $2a-3b=1$ を$b$について解きなさい。%% H27前期, 0205
    \item $b=\bunsuu{2}{3}a-\bunsuu{1}{3}$
    \item $\sqrt{32}-2\sqrt{18}+5\sqrt{2}$を計算しなさい。%% H27前期, 0310
    \item $3\sqrt{2}$
    \item 二次方程式 $x^2-2x=3(x-1)$ を解きなさい。%% H27前期, 0320
    \item $x=\bunsuu{5\pm\sqrt{13}}{2}$
    %% H28後期
    \item $8-15$ を計算しなさい。%% H28後期, 0100
    \item $-7$
    \item $3+(-4)^2\div8$ を計算しなさい。%% H28後期, 0105
    \item $5$
    \item $3\left(\bunsuu{1}{2}x-\bunsuu{2}{3}y\right)-\bunsuu{1}{2}x-y$ を計算しなさい。%% H28後期, 0200
    \item $x-3y$
    \item 連立方程式 $\trenritu{x-2y=-8\\3x+y=11}$ を解きなさい。%% H28後期, 0210
    \item $x=2, y=5$
    \item $-\sqrt{75}+\bunsuu{12}{\sqrt{3}}$ を計算しなさい。%% H28後期, 0310
    \item $-\sqrt{3}$
    \item 二次方程式 $2x^2-3x-1=0$ を解きなさい。%% H28後期, 0320
    \item $x=\bunsuu{3\pm\sqrt{17}}{4}$
    %% H28前期
    \item $-18\div(-3)$ を計算しなさい。%% H28前期, 0100
    \item $6$
    \item $-3^2+16\times\bunsuu{3}{4}$ を計算しなさい。%% H28前期, 0105
    \item $3$
    \item $2x+3y-\bunsuu{x+5y}{2}$ を計算しなさい。%% H28前期, 0200
    \item $\bunsuu{3}{2}x+\bunsuu{1}{2}y$
    \item 方程式 $x+3.5=0.5(3x-1)$ を解きなさい。%% H28前期, 0120
    \item $x=8$
    \item $(\sqrt{2}-\sqrt{5})^2$ を計算しなさい。%% H28前期, 0310
    \item $7-2\sqrt{10}$
    \item $(x+2)(x-6)-9$ を因数分解しなさい。%% H28前期, 0305
    \item $(x+3)(x-7)$
    %% H29後期
    \item $(-5)+(-4)$ を計算しなさい。%% H29後期, 0100
    \item $-9$
    \item $(-4)^2-8\times\bunsuu{3}{2}$ を計算しなさい。%% H29後期, 0105
    \item $4$
    \item $\bunsuu{2x+y}{3}+\bunsuu{x-y}{2}$ を計算しなさい。%% H29後期, 0200
    \item $\bunsuu{7x-y}{6}$
    \item $\bunsuu{6}{\sqrt{2}}+\sqrt{8}$ を計算しなさい。%% H29後期, 0310
    \item $5\sqrt{2}$
    \item 七角形の内角の和を求めなさい。%% H29後期, 0230
    \item $900^\circ$
    \item $x^2-5x-6$ を因数分解しなさい。%% H29後期, 0305
    \item $(x+1)(x-6)$
    %% H29前期
    \item $(-8)\times(-2)$ を計算しなさい。%% H29前期, 0100
    \item $16$
    \item $6-(-2)^2\div\bunsuu{4}{9}$ を計算しなさい。%% H29前期, 0105
    \item $-3$
    \item $2(x-3y)-3(x-4y)$ を計算しなさい。%% H29前期, 0200
    \item $-x+6y$
    \item 等式 $4x-3y=15$ を$y$について解きなさい。%% H29前期, 0205
    \item $y=\bunsuu{4}{3}x-5$
    \item $3\sqrt{5}-\sqrt{80}+\sqrt{20}$ を計算しなさい。%% H29前期, 0310
    \item $\sqrt{5}$
    \item 二次方程式 $3x^2+7x+1=0$ を解きなさい。%% H29前期, 0320
    \item $x=\bunsuu{-7\pm\sqrt{37}}{6}$
    %% 2018後期
    \item $-10-(-4)$ を計算しなさい。%% 2018後期, 0100
    \item $-6$
    \item $6\times\left(-\bunsuu{2}{3}\right)^2$ を計算しなさい。%% 2018後期, 0105
    \item $\bunsuu{8}{3}$
    \item $4(3x-2y)-5(x-2y)$ を計算しなさい。%% 2018後期, 0200
    \item $7x+2y$
    \item $xy^2\div2y\times8x$ を計算しなさい。%% 2018後期, 0200
    \item $4x^2y$
    \item $\sqrt{3}(\sqrt{12}+\sqrt{6})$ を計算しなさい。%% 2018後期, 0310
    \item $6+3\sqrt{2}$
    \item 二次方程式 $2x^2-3x-4=0$ を解きなさい。%% 2018後期, 0320
    \item $x=\bunsuu{3\pm\sqrt{41}}{4}$
    %% 2018前期
    \item $(-4)+(-8)$ を計算しなさい。%% 2018前期, 0100
    \item $-12$
    \item $(-3)^2+12\div(-2)$ を計算しなさい。%% 2018前期, 0105
    \item $3$
    \item $\bunsuu{2}{3}(5a-3b)-3a+4b$ を計算しなさい。%% 2018前期, 0200
    \item $\bunsuu{1}{3}a+2b$
    \item 連立方程式 $\trenritu{2x+3y=9\\y=3x+14}$ を解きなさい。%% 2018前期, 0210
    \item $x=-3, y=5$
    \item $2\sqrt{27}-\bunsuu{6}{\sqrt{3}}$ を計算しなさい。%% 2018前期, 0310
    \item $4\sqrt{3}$
    \item $(x+3)(x-5)+2(x+3)$ を因数分解しなさい。%% 2018前期, 0305
    \item $(x+3)(x-3)$
    %% 2019後期
    \item $12-(-6)$ を計算しなさい。%% 2019後期, 0100
    \item $18$
    \item $-5^2\div\bunsuu{5}{4}$ を計算しなさい。%% 2019後期, 0105
    \item $-20$
    \item $3(2a+b)-5\left(\bunsuu{4}{5}a+\bunsuu{1}{10}b\right)$ を計算しなさい。%% 2019後期, 0200
    \item $2a+\bunsuu{5}{2}b$
    \item 連立方程式 $\trenritu{2x-3y=17\\3x+5y=-3}$ を解きなさい。%% 2019後期, 0210
    \item $x=4, y=-3$
    \item $(\sqrt{7}-\sqrt{3})(\sqrt{7}-2\sqrt{3})$ を計算しなさい。%% 2019後期, 0310
    \item $13-3\sqrt{21}$
    \item $(x+4)(x-3)-8$ を因数分解しなさい。%% 2019後期, 0305
    \item $(x-4)(x+5)$
    %% 2019前期
    \item $15\div(-3)$ を計算しなさい。%% 2019前期, 0100
    \item $-5$
    \item $7-\left(-\bunsuu{3}{4}\right)\times(-2)^2$ を計算しなさい。%% 2019前期, 0105
    \item $10$
    \item $(7x+y)-4\left(\bunsuu{1}{2}x+\bunsuu{3}{4}y\right)$ を計算しなさい。%% 2019前期, 0200
    \item $5x-2y$
    \item 等式 $9a+3b=2$ を$b$について解きなさい。%% 2019前期, 0205
    \item $b=-3a+\bunsuu{2}{3}$
    \item $\bunsuu{4}{\sqrt{2}}-\sqrt{3}\times\sqrt{6}$ を計算しなさい。%% 2019前期, 0310
    \item $-\sqrt{2}$
    \item 二次方程式 $2x^2+x-4=0$ を解きなさい。%% 2019前期, 0320
    \item $x=\bunsuu{-1\pm\sqrt{33}}{4}$
    %% 2020後期
    \item $6\times(-3)$ を計算しなさい。%% 2020後期, 0100
    \item $-18$
    \item $9-(-4)^2\times\bunsuu{5}{8}$ を計算しなさい。%% 2020後期, 0105
    \item $-1$
    \item $a^2b\times21b\div7a$ を計算しなさい。%% 2020後期, 0200
    \item $3ab^2$
    \item 連立方程式 $\trenritu{0.2x+1.5y=4\\x-3y=-1}$ を解きなさい。%% 2020後期, 0210
    \item $x=5, y=2$
    \item $\bunsuu{12}{\sqrt{3}}-3\sqrt{6}\times\sqrt{8}$ を計算しなさい。%% 2020後期, 0310
    \item $-8\sqrt{3}$
    \item 二次方程式 $x^2+5x+5=0$ を解きなさい。%% 2020後期, 0320
    \item $x = \bunsuu{-5\pm\sqrt{5}}{2}$
    %% 2020前期
    \item $-2+9$ を計算しなさい。%% 2020前期, 0100
    \item $7$
    \item $-5^2+18\div\bunsuu{3}{2}$ を計算しなさい。%% 2020前期, 0105
    \item $-13$
    \item $2(x+4y)-3\left(\bunsuu{1}{2}x-\bunsuu{1}{3}y\right)$ を計算しなさい。%% 2020前期, 0200
    \item $\bunsuu{1}{2}x+9y$
    \item 方程式 $x-7=\bunsuu{4x-9}{3}$ を解きなさい。%% 2020前期, 0120
    \item $x=-12$
    \item $\sqrt{50}+6\sqrt{2}-\bunsuu{14}{\sqrt{2}}$ を計算しなさい。%% 2020前期, 0310
    \item $4\sqrt{2}$
    \item $2x^2-32$ を因数分解しなさい。%% 2020前期, 0305
    \item $2(x+4)(x-4)$
    %% 2021
    \item $-5\times(-8)$ を計算しなさい。%% 2021, 0100
    \item $40$
    \item $-9+(-2)^3\times\bunsuu{1}{4}$ を計算しなさい。%% 2021, 0105
    \item $-11$
    \item $(8a-5b)-\bunsuu{1}{3}(6a-9b)$ を計算しなさい。%% 2021, 0200
    \item $6a-2b$
    \item 連立方程式 $\trenritu{2x+3y=7\\3x-y=-17}$ を解きなさい。%% 2021, 0210
    \item $x=-4, y=5$
    \item $\bunsuu{12}{\sqrt{6}}+\sqrt{42}\div\sqrt{7}$ を計算しなさい。%% 2021, 0310
    \item $3\sqrt{6}$
    \item 二次方程式 $x^2+9x+7=0$ を解きなさい。%% 2021, 0320
    \item $x=\bunsuu{-9\pm\sqrt{53}}{2}$
    %%URL https://happylilac.net/jhs-math1_02-01ans.pdf
    \item $b \times a \times 3$ を簡単にしなさい。%% 1-3-1, 0110
    \item $3ab$
    \item $y \times 0.1$ を簡単にしなさい。%% 1-3-2, 0110
    \item $0.1y$
    \item $5-a \times b \times a$ を簡単にしなさい。%% 1-3-3, 0110
    \item $5-a^2b$
    \item $a \div 3 \times b$ を簡単にしなさい。%% 1-3-4, 0110
    \item $\bunsuu{ab}{3}$
    \item $a \times (-3) + 1$ を簡単にしなさい。%% 2-1-1, 0110
    \item $-3a+1$
    \item $y \times y \times y \times (-z) \times (-z)$ を簡単にしなさい。%% 2-1-2, 0110
    \item $y^3z^2$
    \item $a \times 2 \div c \times b$ を簡単にしなさい。%% 2-1-3, 0110
    \item $\bunsuu{2ab}{c}$
    \item $7 \div (a - 1) \times b$ を簡単にしなさい。%% 2-1-4, 0110
    \item $\bunsuu{7b}{a-1}$
    \item $b \div a \times (-5)$ を簡単にしなさい。%% 3-1-1, 0110
    \item $-\bunsuu{5b}{a}$
    \item $11-a\times(-2)$ を簡単にしなさい。%% 3-1-2, 0110
    \item $11+2a$
    \item $0.1 \times a + b \times b$ を簡単にしなさい。%% 3-1-3, 0110
    \item $0.1a+b^2$
    \item $(a - 4) \div 8 \times b$ を簡単にしなさい。%% 3-1-4, 0110
    \item $\bunsuu{(a-4)b}{b}$
    \item $8 \div a \div 3$ を簡単にしなさい。%% 4-1-1, 0110
    \item $\bunsuu{8}{3a}$
    \item $a \times (-1) + b \times 7$ を簡単にしなさい。%% 4-1-2, 0110
    \item $-a+7b$
    \item $5 \div a \times b \times b$ を簡単にしなさい。%% 4-1-3, 0110
    \item $\bunsuu{5b^2}{a}$
    \item $(b - 1) \div 2 \div a$ を簡単にしなさい。%% 4-1-4, 0110
    \item $\bunsuu{b-1}{2a}$
    \item $a=2$のとき,$-3a$ の値を求めなさい。%% 2-4-1, 0115
    \item $-6$
    \item $a=5, b=-2$のとき,$6a+4b$ の値を求めなさい。%% 3-4-1, 0115
    \item $22$
    \item $a=-4, b=3$のとき,$\bunsuu{a}{2}+\bunsuu{6}{b}$ の値を求めなさい。%% 4-4-3, 0115
    \item $0$
    \item $a=8, b=-5$のとき,$2(1-b)^2$ の値を求めなさい。%% 5-4-2, 0115
    \item $72$
    %%URL https://happylilac.net/jhs-math1_02-02ans.pdf
    \item $-2(7x-4)+6(x-3)$ を計算しなさい。%% 5-3-1, 0116
    \item $-8x-10$
    \item $\bunsuu{1}{3}(9x+6)-\bunsuu{1}{4}(12-8x)$ を計算しなさい。%% 5-3-2, 0116
    \item $5x-1$
    \item $(28x -8)\div4$ を計算しなさい。%% 5-2-5, 0116
    \item $7x-2$
    \item $(25x -10)\div(-5)$ を計算しなさい。%% 5-2-6, 0116
    \item $-5x+2$
    \item $-\bunsuu{3}{2}(4x-6)$ を計算しなさい。%% 5-2-3, 0116
    \item $-6x+9$
    \item $12\times\bunsuu{4-x}{6}$ を計算しなさい。%% 5-2-2, 0116
    \item $8-2x$
    \item $4(4x-5)$ を計算しなさい。%% 5-2-1, 0116
    \item $16x-20$
    \item $(7-6x)\times(-5)$ を計算しなさい。%% 5-2-2, 0116
    \item $-35+30x$
    \item $(3x+10)-(6x-5)$ を計算しなさい。%% 5-1-5, 0116
    \item $-3x+15$
    \item $(7x-12)-(6x-11)$ を計算しなさい。%% 5-1-6, 0116
    \item $x-1$
    \item $(5x-3)+(7x-2)$ を計算しなさい。%% 5-1-3, 0116
    \item $12x-5$
    \item $(3x+2)+(6x-5)$ を計算しなさい。%% 5-1-4, 0116
    \item $9x-3$
    \item $(7x+5)-(4x-1)$ を計算しなさい。%% 5-1-1, 0116
    \item $3x+6$
    \item $(5y-6)-(1-8y)$ を計算しなさい。%% 5-1-2, 0116
    \item $13y-7$
    \item $3(3x+4)+3(x-1)$ を計算しなさい。%% 4-3-1, 0116
    \item $12x+9$
    \item $\bunsuu{1}{2}(2x+8)-2(5x+3)$ を計算しなさい。%% 4-3-2, 0116
    \item $-9x-2$
    \item $(9x-15)\div3$ を計算しなさい。%% 4-2-5, 0116
    \item $3x-5$
    \item $(-24x+18)\div(-6)$ を計算しなさい。%% 4-2-6, 0116
    \item $4x-3$
    \item $\bunsuu{1}{3}(6x+3)$ を計算しなさい。%% 4-2-3, 0116
    \item $2x+1$
    \item $\bunsuu{x+2}{5}\times10$ を計算しなさい。%% 4-2-4, 0116
    \item $2x+4$
    %%URL https://math.005net.com/1/dainyu.php
    \item $x=-2$のとき,$\bunsuu{1}{2}x+4$ の値を求めなさい。%% 1-3, 0115
    \item $3$
    \item $x=6$のとき,$2x^2$ の値を求めなさい。%% 2-3, 0115
    \item $72$
    \item $x=-3$のとき,$x^2+x-6$ の値を求めなさい。%% 3-1, 0115
    \item $0$
    \item $x=\bunsuu{3}{2}$のとき,$\bunsuu{3}{x}$ の値を求めなさい。%% 4-2, 0115
    \item $2$
    \item $a=2,b=-3$のとき,$5a-2b$ の値を求めなさい。%% 5-1, 0115
    \item $16$
    %%URL https://www.gyaku10study.net/math_tr/math1-tr10.php
    \item $a=-2$のとき,$a^2$ の値を求めなさい。%% 1-4, 0115
    \item $4$
    \item $x=-5$のとき,$3x+4$ の値を求めなさい。%% 2-2, 0115
    \item $-11$
    \item $a=3, b=-4$のとき,$5a+2b$ の値を求めなさい。%% 3-1, 0115
    \item $7$
    %%URL https://chugaku.manabihiroba.net/wp/wp-content/uploads/2016/05/sikinoatai1_1.pdf
    \item $a=5$のとき,$2a-(-8a+4)$ の値を求めなさい。%% 4-1, 0115
    \item $46$
    \item $a=2, b=-4$のとき,$2ab$ の値を求めなさい。%% 5-1, 0115
    \item $-16$
    %%URL https://chugaku.manabihiroba.net/wp/wp-content/uploads/2016/05/sikinoatai1_2.pdf
    \item $x=-3$のとき,$3x-2$ の値を求めなさい。%% 2-1, 0115
    \item $-11$
    \item $x=-2$のとき,$2x^3-8$ の値を求めなさい。%% 3-3, 0115
    \item $-24$
    \item $a=4$のとき,$3(a+2)-6(3a-2)$ の値を求めなさい。%% 4-2, 0115
    \item $-42$
    %%URL https://examist.jp/junior-math/number-expression/dainyuu-sikinoatai/
    \item $x=-3$のとき,$4x-5$ の値を求めなさい。%% 1-1, 0115
    \item $-17$
    \item $x=-\bunsuu{3}{2}$のとき,$\bunsuu{3}{4}-\bunsuu{5}{2}x$ の値を求めなさい。%% 2-1, 0115
    \item $\bunsuu{9}{2}$
    %%URL https://happylilac.net/pdf/pg0014-001ans-01.pdf
    \item $-7+(-5)$ を計算しなさい。%% 1-1, 0100
    \item $-12$
    \item $(-6)-(+9)$ を計算しなさい。%% 1-2, 0100
    \item $-15$
    \item $6\div15\times(-20)$ を計算しなさい。%% 1-10, 0105
    \item $-8$
    \item $15-(-21)\div\left(-\bunsuu{3}{4}\right)$ を計算しなさい。%% 1-14, 0105
    \item $-13$
    \item $-3^2+12\div(-4)$ を計算しなさい。%% 19-13, 0105
    \item $-12$
    \item $(-5)\times3^2+(-2)^3$ を計算しなさい。%% 19-15, 0105
    \item $-53$
    %%URL https://examist.jp/junior-math/number-expression/mojisiki/
    \item $\bunsuu{2a^2b}{c}$ を$\times,\div$の記号を用いて表せ。%% 2-1, 0110
    \item $2\times a \times a \times b \div c$
    \item $\bunsuu{3z}{x+y}$ を$\times,\div$の記号を用いて表せ。%% 2-2, 0110
    \item $3\times z \div (x+y)$
    \item $\bunsuu{a+b}{5}$ を$\times,\div$の記号を用いて表せ。%% 1-4, 0110
    \item $(a+b)\div5$
    \item $6\div(a+b)\times5$ を簡単にしなさい。%% 1-6, 0110
    \item $\bunsuu{30}{a+b}$
    %%URL https://examist.jp/junior-math/equation/itijihouteisiki/
    \item 方程式 $x-4=3$ を解きなさい。%% 1-1, 0120
    \item $x=7$
    \item 方程式 $3x=-9$ を解きなさい。%% 1-2, 0120
    \item $x=-3$
    \item 方程式 $2x-5=7$ を解きなさい。%% 1-3, 0120
    \item $x=6$
    \item 方程式 $3x-8=5x-4$ を解きなさい。%% 1-6, 0120
    \item $x=-2$
    \item 方程式 $5-4x=9$ を解きなさい。%% 1-5, 0120
    \item $x=-1$
    %%URL https://happylilac.net/jhs-math2_01-03ans.pdf
    \item 等式 $\ell=2\pi r$ を$r$について解きなさい。%% 5-3-4, 0205
    \item $r=\bunsuu{\ell}{2\pi}$
    \item 等式 $y=4x-5$ を$x$について解きなさい。%% 5-3-1, 0205
    \item $x=\bunsuu{y+5}{4}$
    \item 等式 $5x-4y=8$ を$y$について解きなさい。%% 4-3-2, 0205
    \item $y=\bunsuu{5x-8}{4}$
    \item 等式 $\ell=2(a+b)$ を$a$について解きなさい。%% 4-3-3, 0205
    \item $a=\bunsuu{\ell-2b}{2}$
    \item 等式 $m=\bunsuu{a+b}{2}$ を$b$について解きなさい。%% 4-3-4, 0205
    \item $b=2m-a$
    \item 等式 $2x+y=5$ を$x$について解きなさい。%% 3-1-1, 0205
    \item $x=\bunsuu{5-y}{2}$
    \item 等式 $4x-y=3$ を$y$について解きなさい。%% 3-1-3, 0205
    \item $y=4x-3$
    \item 等式 $7xy+5 = 0$ を$y$について解きなさい。%% 3-1-4, 0205
    \item $y=-\bunsuu{5}{7x}$
    %%URL https://examist.jp/junior-math/number-expression/tousiki-henkei/
    \item 等式 $V-\bunsuu{1}{3}xyz$ を$z$について解きなさい。%% 1-2, 0205
    \item $z=\bunsuu{3V}{xy}$
    %%URL https://happylilac.net/jhs-math2_02-01ans.pdf
    \item 連立方程式 $\trenritu{2x-y=1\\3x-y=5}$ を解きなさい。%% 2-1-2, 0210
    \item $x=4, y=7$
    \item 連立方程式 $\trenritu{x-2y=5\\2x+3y=-4}$ を解きなさい。%% 2-1-3, 0210
    \item $x=1, y=-2$
    \item 連立方程式 $\trenritu{2x-y=9\\3x+4y=8}$ を解きなさい。%% 2-1-4, 0210
    \item $x=4, y=-1$
    \item 連立方程式 $\trenritu{4x-3y=-5\\y=3x}$ を解きなさい。%% 3-1-1, 0210
    \item $x=1, y=3$
    \item 連立方程式 $\trenritu{2x+y=12\\x=-2y}$ を解きなさい。%% 3-1-2, 0210
    \item $x=8, y=-4$
    \item 連立方程式 $\trenritu{3x-2y=7\\x=y+4}$ を解きなさい。%% 3-1-3, 0210
    \item $x=-1, y=-5$
    \item 連立方程式 $\trenritu{2x+3y=13\\y=2x-1}$ を解きなさい。%% 3-1-4, 0210
    \item $x=2, y=3$
    \item 連立方程式 $\trenritu{5x+3y=4\\4x-3y=14}$ を解きなさい。%% 4-1-1, 0210
    \item $x=2, y=-2$
    \item 連立方程式 $\trenritu{-2x+3y=17\\5x+9y=7}$ を解きなさい。%% 4-1-2, 0210
    \item $x=-4, y=3$
    \item 連立方程式 $\trenritu{2x-3y=-8\\3x-4y=-9}$ を解きなさい。%% 4-1-3, 0210
    \item $x=5, y=6$
    %%URL https://happylilac.net/jhs-math2_02-02ans.pdf
    \item 連立方程式 $\trenritu{2(x+y)-3y=6\\x+y=6}$ を解きなさい。%% 1-1-1, 0210
    \item $x=4, y=2$
    \item 連立方程式 $\trenritu{3x-5(x-y)=-22\\2x+y=10}$ を解きなさい。%% 1-1-2, 0210
    \item $x=6, y=-2$
    \item 連立方程式 $\trenritu{\bunsuu{3}{8}x+\bunsuu{y}{4}=2\\x-y=-3}$ を解きなさい。%% 1-1-3, 0210
    \item $x=2, y=5$
    \item 連立方程式 $\trenritu{0.5x+0.6y=1.3\\x+3y=8}$ を解きなさい。%% 1-1-4, 0210
    \item $x=-1, y=3$
    \item 方程式 $2x+y = -x+3y = 7$ を解きなさい。%% 2-1-1, 0210
    \item $x=2, y=3$
    %%URL https://happylilac.net/pdf/pg0014-003ans.pdf
    \item $2a(a-2)+3a(1+3a)$ を計算しなさい。%% 1-2, 0300
    \item $11a^2-a$
    \item $(6x^2-9x)\div(-3x)$ を計算しなさい。%% 1-8, 0300
    \item $-2x+3$
    \item $(x+3)(x+7)$ を展開しなさい。%% 1-9, 0300
    \item $x^2+10x+21$
    \item $(4a-8b)(4a+8b)$ を展開しなさい。%% 1-10, 0300
    \item $16a^2-64b^2$
    \item $(x-3)^2$ を展開しなさい。%% 1-11, 0300
    \item $x^2-6x+9$
    \item $(x+3)^2+(x-12)(x+3)$ を計算しなさい。%% 1-12, 0300
    \item $2x^2-3x-27$
    \item $3a(a-4)+2a(a+3)$ を計算しなさい。%% 2-4, 0300
    \item $5a^2-6a$
    \item $(x-5)(y+3)$ を展開しなさい。%% 3-6, 0300
    \item $xy+3x-5y-15$
    \item $(3x+1)(3x-1)$ を展開しなさい。%% 3-10, 0300
    \item $9x^2-1$
    %%URL https://happylilac.net/jhs-math3_01-01ans.pdf
    \item $\bunsuu{5}{2}(4x+14y)$ を計算しなさい。%% 4-1-1, 0300
    \item $10x^2+35xy$
    \item $(8x+4y)\times\bunsuu{3}{2}y$ を計算しなさい。%% 4-1-2, 0300
    \item $12xy+6y^2$
    \item $(6xy-10y^2)\div\bunsuu{2}{5}y$ を計算しなさい。%% 4-1-3, 0300
    \item $15x-25y$
    \item $(9x^2y+12xy^3)\div3xy$ を計算しなさい。%% 4-1-4, 0300
    \item $3x+4y$
    \item $(24x^2-8xy)\div(-4x)$ を計算しなさい。%% 4-1-5, 0300
    \item $-6x+2y$
    \item $(x+y-4)(x+y+7)$ を展開しなさい。%% 4-2-10, 0300
    \item $x^2+2xy+y^2+3x+3y-28$
    \item $(x-3y)(x+5y)$ を展開しなさい。%% 4-2-9, 0300
    \item $x^2+2xy-15y^2$
    %%URL https://happylilac.net/jhs-math3_01-02ans.pdf
    \item $ax+bx$を因数分解しなさい。%% 1-1-1, 0305
    \item $x(a+b)$
    \item $x^2-2x$を因数分解しなさい。%% 1-1-5, 0305
    \item $x(x-2)$
    \item $2x^2y-3xy^2+xy$を因数分解しなさい。%% 1-1-10, 0305
    \item $xy(2x-3y+1)$
    \item $x^2+8x+7$を因数分解しなさい。%% 2-1-1, 0305
    \item $(x+1)(x+7)$
    \item $x^2-5x+6$を因数分解しなさい。%% 2-1-2, 0305
    \item $(x-2)(x+3)$
    \item $x^2+3x-18$を因数分解しなさい。%% 2-1-3, 0305
    \item $(x-3)(x+6)$
    \item $x^2-5x-36$を因数分解しなさい。%% 2-1-4, 0305
    \item $(x-9)(x+4)$
    \item $x^2+4x+4$を因数分解しなさい。%% 2-1-5, 0305
    \item $(x+2)^2$
    \item $x^2-10x+25$を因数分解しなさい。%% 2-1-6, 0305
    \item $(x-5)^2$
    \item $x^2-9$を因数分解しなさい。%% 2-1-7, 0305
    \item $(x+3)(x-3)$
    \item $x^2-49$を因数分解しなさい。%% 2-1-8, 0305
    \item $(x+7)(x-7)$
    \item $(x+1)^2-16$を因数分解しなさい。%% 4-1-4, 0305
    \item $(x+5)(x-3)$
    \item $81-x^2$を因数分解しなさい。%% 5-1-6, 0305
    \item $(9+x)(9-x)$
    \item $(x+3)^2-8(x+3)+16$を因数分解しなさい。%% 6-2-1, 0305
    \item $(x-1)^2$
    \item $9x^2-12x+4$を因数分解しなさい。%% 4-1-3, 0305
    \item $(3x-2)^2$
    \item $2x^2y+12xy+18y$を因数分解しなさい。%% 4-1-1, 0305
    \item $2y(x+3)^2$
    %%URL https://happylilac.net/jhs-math2_01-02ans.pdf
    \item $6x^2\times 2y\div 3x$ を計算しなさい。%% 2-2-1, 0200
    \item $4xy$
    \item $9y^3\div3y\times4x$ を計算しなさい。%% 4-2-7, 0200
    \item $12xy^2$



  \end{enumerate}
\newpage
\end{multicols}
\end{document}
